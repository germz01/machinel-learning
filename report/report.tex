\documentclass[11pt,twoside]{article}

\usepackage{blindtext}
\usepackage[T1]{fontenc}
\usepackage{microtype}
\usepackage[english]{babel}
\usepackage[hmarginratio=1:1,top=32mm,columnsep=20pt]{geometry}
\usepackage[hang, small,labelfont=bf,up,textfont=it,up]{caption}
\usepackage{booktabs}
\usepackage{lettrine}
\usepackage{enumitem}
\setlist[itemize]{noitemsep}

\usepackage{abstract}
\renewcommand{\abstractnamefont}{\normalfont\bfseries}
\renewcommand{\abstracttextfont}{\normalfont\small}

\usepackage{titlesec}
\titleformat{\section}[block]{\large\scshape\centering}{\thesection.}{1em}{}
\titleformat{\subsection}[block]{\large}{\thesubsection.}{1em}{}
\usepackage{titling}
\usepackage{hyperref}

%----------------------------------------------------------------------------------------
%   TITLE SECTION
%----------------------------------------------------------------------------------------

\setlength{\droptitle}{-4\baselineskip} % Move the title up

\pretitle{\begin{center}\Huge\bfseries} % Article title formatting
\posttitle{\end{center}} % Article title closing formatting
\title{Machine Learning} % Article title
\author{%
\textsc{Gianmarco Ricciarelli} \\[1ex] % Your name
\normalsize \href{mailto:john@smith.com}{gianmarcoricciarelli@gmail.com} % Your email address
\and % Uncomment if 2 authors are required, duplicate these 4 lines if more
\textsc{Stefano Carpita} \\[1ex] % Second author's name
\normalsize \href{mailto:jane@smith.com}{carpitastefano@gmail.com} % Second author's email address
}
\date{
    ML - Academic Year: 2018/2019 \\
    \today \\
    Type of project: \textbf{A}
}
\renewcommand{\maketitlehookd}{%
\begin{abstract}
\noindent With this report we describe our type \textbf{A} project for the Machine Learning course. We experiment
two datasets, namely MONK's and CUP, by building from scratch a neural network's implementation, which we've
validated by searching for the best hyper-parameters' combination via well-known validation techniques. Finally,
for each one of the datasets, we collected the data describing the results.
\end{abstract}
}

%----------------------------------------------------------------------------------------

\begin{document}

% Print the title
\maketitle

%----------------------------------------------------------------------------------------
%   ARTICLE CONTENTS
%----------------------------------------------------------------------------------------

\section{Introduction} % (fold)
\label{sec:introduction}
    By chosing the type \textbf{A} project, we followed the goal of optimizing a built from scratch neural
    network model by searching the best hyper-parameters' combination in order to obtain the best
    performances on the MONK's and CUP datasets. The model we've built implements the
    standard backpropagation algorithm with gradient descent, as described in \cite{deep_learning}. As
    optimization tool, we've searched the so-called hyper-parameters space by using well-known validation
    algorithms like k-fold cross validation and grid search, described in \cite{deep_learning} and
    \cite{random_search}. More technical details can be founded in section \ref{sec:methods}.
% section introduction (end)

%------------------------------------------------

\section{Methods} % (fold)
\label{sec:methods}
    Our code base is written using the \textit{Python} programming language, which was enhanced with
    numerical libraries like \textit{NumPy} and \textit{Pandas} to ease the computational cost involved in
    implementating popular machine learning algorithms. Also the standard library function provided by the
    language were utilized. Following the philosophy of type \textbf{A} projects, we haven't used
    out-of-the-box implementations for the algorithms composing our framework.

    \subsection{Code overview} % (fold)
    \label{sub:code_overview}
        We can view our code base divided in two essential branches: the \textit{neural network} branch and
        the \textit{model selection and assessment} branch. The first one consist in the code implementing the
        neural network model, which is contained in a \texttt{NeuralNetwork} class, and all the code related
        to the building and training of such a model, e.g. the activation functions, the losses, the
        regularization metrics and so on, that is divided in several scripts in order to ease the implementation.
        The code base's second branch consist in the model selection and assessment related code, which is
        composed by several classes, each one implementing one algorithm, like \texttt{GridSearch},
        \texttt{KFold\_Cross\_validation} and so on, utilized for the model's validation, selection and
        assessment. The second branch also is splitted in several scripts.
    % subsection code_overview (end)

    \subsection{Implementation choices} % (fold)
    \label{sub:implementation_choices}
        Since we wanted to be able to take different paths during the building phase of our model, we expanded
        the first code base's branch by writing diffent activation functions, losses and regularization metrics.
        In the \texttt{activation} script we writed the code for the \texttt{sigmoid}, \texttt{relu},
        \texttt{tanh} and \texttt{identity} activation functions. The \texttt{regularizers} script contains the
        code for implementing both the L1 and L2 regularization metrics, as described in \cite{deep_learning},
        and the \texttt{losses} script contains the code for implementing popular loss function like, for
        example, the \texttt{mean\_squared\_error} and the \texttt{mean\_euclidean\_error}. Al the metrics
        described so far are used in the \texttt{nn} script, which contains the code for implementing the neural
        network model, and hence the backpropagation and training algorithm. The \texttt{NeuralNetwork} class
        represents a neural network using the standard backpropagation algorithm with gradient descent,
        standard momentum and regularization. The \texttt{NeuralNetwork} class contains also the code for
        implementing the early stopping metrics, which we implemented following \cite{early_stopping}.
        More details about what particular combination of metrics was used during the experimental phase can be
        found in section \ref{sec:experiments}.
    % subsection implementation_choices (end)
% section methods (end)

%------------------------------------------------

\section{Experiments} % (fold)
\label{sec:experiments}

% section experiments (end)

%------------------------------------------------

\section{Conclusions}

%----------------------------------------------------------------------------------------
%   REFERENCE LIST
%----------------------------------------------------------------------------------------

\begin{thebibliography}{99} % Bibliography - this is intentionally simple in this template
    \bibitem{deep_learning}
    Ian Goodfellow, Yoshua Bengio and Aaron Courville.
    \textit{Deep Learning}, MIT Press, 2016.

    \bibitem{random_search}
    James Bergstra and Yoshua Bengio.
    \textit{Random Search for Hyper-parameter Optimization}, J. Mach. Learn. Res. 13, pp. 281-305, 2012.

    \bibitem{early_stopping}
    Prechelt L..
    \textit{Early Stopping - But When?}, Montavon G., Orr G.B., Müller KR. (eds) Neural Networks: Tricks of the Trade. Lecture Notes in Computer Science, vol 7700. Springer, Berlin, Heidelberg, 2012.
\end{thebibliography}

%----------------------------------------------------------------------------------------

\end{document}
