\documentclass[11pt,twoside]{article}

\usepackage{blindtext}
\usepackage[T1]{fontenc}
\usepackage{microtype}
\usepackage[english]{babel}
\usepackage[hmarginratio=1:1,top=32mm,columnsep=20pt]{geometry}
\usepackage[hang, small,labelfont=bf,up,textfont=it,up]{caption}
\usepackage{booktabs}
\usepackage{lettrine}
\usepackage{enumitem}
\setlist[itemize]{noitemsep}

\usepackage{abstract}
\renewcommand{\abstractnamefont}{\normalfont\bfseries}
\renewcommand{\abstracttextfont}{\normalfont\small}

\usepackage{titlesec}
\titleformat{\section}[block]{\large\scshape\centering}{\thesection.}{1em}{}
\titleformat{\subsection}[block]{\large}{\thesubsection.}{1em}{}
\usepackage{titling}
\usepackage{hyperref}

%----------------------------------------------------------------------------------------
%   TITLE SECTION
%----------------------------------------------------------------------------------------

\setlength{\droptitle}{-4\baselineskip} % Move the title up

\pretitle{\begin{center}\Huge\bfseries} % Article title formatting
\posttitle{\end{center}} % Article title closing formatting
\title{Machine Learning} % Article title
\author{%
\textsc{Gianmarco Ricciarelli} \\[1ex] % Your name
\normalsize \href{mailto:john@smith.com}{gianmarcoricciarelli@gmail.com} % Your email address
\and % Uncomment if 2 authors are required, duplicate these 4 lines if more
\textsc{Stefano Carpita} \\[1ex] % Second author's name
\normalsize \href{mailto:jane@smith.com}{carpitastefano@gmail.com} % Second author's email address
}
\date{
    ML - Academic Year: 2018/2019 \\
    \today \\
    Type of project: \textbf{A}
}
\renewcommand{\maketitlehookd}{%
\begin{abstract}
\noindent With this report we describe our type \textbf{A} project for the Machine Learning course. We experiment
two datasets, namely MONK's and CUP, by building from scratch a neural network's implementation, which we've
validated by searching for the best hyperparameters' combination via well-known validation techniques. Finally,
for each one of the datasets, we collected the data describing the results.
\end{abstract}
}

%----------------------------------------------------------------------------------------

\begin{document}

% Print the title
\maketitle

%----------------------------------------------------------------------------------------
%   ARTICLE CONTENTS
%----------------------------------------------------------------------------------------

\section{Introduction} % (fold)
\label{sec:introduction}
    By chosing the type \textbf{A} project, we followed the goal of optimizing a built from scratch neural
    network model by searching the best hyperparameters' combination in order to obtain the best
    performances on the MONK's and CUP datasets. The model we've built implements the
    standard backpropagation algorithm with gradient descent, as described in \cite{deep_learning}. As
    optimization tool, we've searched the so-called hyperparameters space by using well-known validation
    algorithms like k-fold cross validation and grid search, also described in \cite{deep_learning}. More
    technical details can be founded in section \ref{sec:methods}.
% section introduction (end)

%------------------------------------------------

\section{Methods} % (fold)
\label{sec:methods}

% section methods (end)

%------------------------------------------------

\section{Experiments}

%------------------------------------------------

\section{Conclusions}

%----------------------------------------------------------------------------------------
%   REFERENCE LIST
%----------------------------------------------------------------------------------------

\begin{thebibliography}{99} % Bibliography - this is intentionally simple in this template
    \bibitem{deep_learning}
    Ian Goodfellow, Yoshua Bengio and Aaron Courville.
    \textit{Deep Learning}, MIT Press, 2016.
\end{thebibliography}

%----------------------------------------------------------------------------------------

\end{document}
